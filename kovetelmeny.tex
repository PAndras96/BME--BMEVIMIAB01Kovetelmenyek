\documentclass[a4paper,12pt]{article}
\usepackage{t1enc}
\usepackage[utf8]{inputenc}
\usepackage[magyar]{babel}
\usepackage{geometry}
\usepackage{tikz}
\usetikzlibrary{calc}
\usepackage{makecell}
\usepackage{pdflscape}
\usepackage[hidelinks]{hyperref}

\geometry{left=20mm,right=20mm,top=20mm,bottom=20mm}
\setlength{\parindent}{0pt}
\setlength{\parskip}{6pt}

\begin{document}
	\begin{center}
		\LARGE Méréstechnika (BMEVIMIAB01) követelményrendszer
	\end{center}
	Az alábbiakban a Méréstechnika (BMEVIMIAB01) tárgy követelményrendszerének nemhivatalos leírása, szemléltetése található. \textbf{Figyelem:} jelen dokumentum csupán szemléltetés, amennyiben a tantárgyi adatlapról (\url{https://portal.vik.bme.hu/kepzes/targyak/VIMIAB01/}), a TVSZ-ből és/vagy az oktatóktól más információ származik, azokat a forrásokat kell érvényesnek tekinteni.
	
	Jelenlét
	\begin{itemize}
		\item Előadás: nem kötelező
		\item Gyakorlat: a gyakorlatok $70\%$-án való részvétel kötelező (ez $14$ gyakorlat esetén $10$ alkalmat jelent)
	\end{itemize}
	Nagy zárthelyi (NZH)
	\begin{itemize}
		\item Egy darab, félév végén ZH-sávban
		\item Elérhető: $20$ pont (jellemzően $2\times5$ pont $+5\times2$ pont)
		\item Követelmény: $8$ pont ($40\%$)
		\item Két alkalommal pótolható, egyszer a szorgalmi időszakban, egyszer a pótlási héten
	\end{itemize}
	Kis zárthelyik (kZH-k)
	\begin{itemize}
		\item Öt darab, félév közben gyakorlatokon
		\item Elérhető: $5\times4$ pont
		\item Követelmény: $3\times2$ pont (vagyis legalább károm kZH-t legalább $50\%$-ra teljesíteni kell)
		\item Ha $4$ kZH legalább $2$ pontos, az $+1$, ha mind az $5$ legalább $2$ pontos, az $+2$ pluszpontot jelent
		\item kZH-k rendje
		\begin{enumerate}
			\item Előző gyakorlaton bejelentés
			\item Adott gyakorlaton kZH megírása
			\item Néhány napon belül az eredmény felkerül a kari Moodle rendszerbe (\url{https://edu.vik.bme.hu})
			\item Következő gyakorlaton megtekintés, reklamáció (a következő gyakorlat alatt az első olyant kell érteni ,amin a hallgató az adott kZH írása után először jelen van)
			\item Ha a hallgató a következő gyakorlaton nem reklamál, az azt jelenti, hogy az eredményét elfogadta
		\end{enumerate}
	\end{itemize}
	Házi feladatok (HF)
	\begin{itemize}
		\item Szorgalmi jellegűek, nem kötelezők
		\item Pluszpontokat lehet velük szerezni
		\item A tárgyhonlapról tölthetők (majd) le (\url{https://www.mit.bme.hu/oktatas/targyak/vimiab01/feladatok}) (majd a feladatok megoldása is itt lesz elérhető)
		\item A feladatok mindenkinek azonosak
		\item Öt darab feladat van a félév során
		\item Határidő: jellemzően péntek este
		\item Leadás: tanszéken (IE 4.emelet) kihelyezett dobozba
		\item A megoldást gyakorlaton egy jó megoldó ismerteti kb. 5 percben, ehhez vázlatként használhatja a megoldását (ha ilyen nincs, akkor pedig a gyakorlatvezető)
		\item Ha valakiről kiderül, hogy másolt, akkor a házi feladatokból egyáltalán nem szerezhet pluszpontokat
		\item A házi feladatokat az előadó javítja, a reklamáció is nála történik
		\item Minden házi feladaton $10$ pontot lehet elérni
		\item A házi feladatokon elért egyéni összpontszám (maximum $50$), valamint az évfolyam házi feladatokon nyújtott teljesítménye alapján legfeljebb $6$ pluszpont szerezhető
	\end{itemize}
	Pluszpontok
	\begin{itemize}
		\item $1$-$1$ pluszpont szerezhető, a kötelező három teljesített (legalább $2$ pontos) kZH-n felüli további egy vagy két teljesített (legalább $2$ pontos) kZH-ért
		\item Maximum $6$ pluszpont szerezhető a házi feladatokból
		\item A legalább elégséges félévközi jegyet a pluszpontok legfeljebb egy jeggyel javíthatják, ehhez $6$ pluszpont szükséges
	\end{itemize}
	Félévközi jegy
	\begin{itemize}
		\item A tárgy félévközi jeggyel zárul
		\item <Féléves pont> $=$ <NZH pont> $+\frac{5}{3}\times$ <három legjobb kZH pontjainak összege>
		\item Így a féléves pont maximum $40$ lehet, erre $40$-$55$-$70$-$85$ százalékos, vagyis $16$-$22$-$28$-$34$ pontos alsó határokat alkalmazva kapható meg a félévközi jegy
		\item Ha a féléves pont legalább $16$, akkor a félévközi jegy számításakor a legfeljebb $6$ plusz ponttal növelt értékét kell figyelembe venni
		\item A NZH és kZH minimumkövetelmények teljesítése elegendő ahhoz, hogy a féléves pont alapján meglegyen az elégséges osztályzat
	\end{itemize}
	\Az{\ref{fig:pontrendszer}}.~ábra a tantárgy követelményrendszerét szemlélteti.
	\begin{landscape}
		\begin{figure}[h]
			% !TeX root = kovetelmeny.tex
\newcommand{\pontMeret}{0.4}
\newcommand{\nagyOsztas}{0.8}
\newcommand{\kisOsztas}{0.4}
\newcommand{\jegyMagassag}{0.8}
\newcommand{\kisTengelyEltolas}{1.2}
\newcommand{\nagyTengelyEltolas}{2}

\newcommand{\pontMarker}[5] %x0,y0,h,color,pont
{\draw[#4] (#1,#2)--++(0,#3) node[anchor=south]{#5} ++(0,-0.3)--++(0.4,0)--++(-0.1,-0.1)++(0.1,0.1)--++(-0.1,0.1);}
\newcommand{\jegyMarker}[6] %x0,y0,h,color,pont,jegy
{\draw[#4] (#1,#2)--++(0,#3) node[anchor=south]{#5} ++(0,-0.3)--++(0.6,0)--++(-0.1,-0.1)++(0.1,0.1)--++(-0.1,0.1) node[anchor=south,draw,yshift=0.5mm]{#6};}
\newcommand{\jegyMarkerAlt}[6] %x0,y0,h,color,pont,jegy
{\draw[#4] (#1,#2)--++(0,#3) node[anchor=south,draw,yshift=0.5mm]{#6} ++(0,-0.3) node[anchor=east]{#5} --++(0.4,0)--++(-0.1,-0.1)++(0.1,0.1)--++(-0.1,0.1);}

\definecolor{darkgreen}{rgb}{0,0.7,0}
\begin{tikzpicture}
	% fő pontszám tengely és osztásai
	\draw (0,0) node[anchor=east]{\makecell[r]{{\color{red}Minimumkövetelmény}\\Szerezhető pont\\{\color{darkgreen}NZH, kHZ minimális járuléka}}} --(40*\pontMeret,0)
	(0,0)--++(0,\nagyOsztas) node[anchor=west]{$0$}
	(\pontMeret,0)--++(0,\kisOsztas)
	(2*\pontMeret,0)--++(0,\kisOsztas)
	(3*\pontMeret,0)--++(0,\kisOsztas)
	(4*\pontMeret,0)--++(0,\kisOsztas)
	(5*\pontMeret,0)--++(0,\nagyOsztas) node[anchor=west]{$5$}
	(6*\pontMeret,0)--++(0,\kisOsztas)
	(7*\pontMeret,0)--++(0,\kisOsztas)
	(8*\pontMeret,0)--++(0,\kisOsztas)
	(9*\pontMeret,0)--++(0,\kisOsztas)
	(10*\pontMeret,0)--++(0,\nagyOsztas) node[anchor=west]{$10$}
	(11*\pontMeret,0)--++(0,\kisOsztas)
	(12*\pontMeret,0)--++(0,\kisOsztas)
	(13*\pontMeret,0)--++(0,\kisOsztas)
	(14*\pontMeret,0)--++(0,\kisOsztas)
	(15*\pontMeret,0)--++(0,\nagyOsztas) node[anchor=south]{$15$}
	(16*\pontMeret,0)--++(0,\kisOsztas)
	(17*\pontMeret,0)--++(0,\kisOsztas)
	(18*\pontMeret,0)--++(0,\kisOsztas)
	(19*\pontMeret,0)--++(0,\kisOsztas)
	(20*\pontMeret,0)--++(0,\nagyOsztas) node[anchor=west]{$20$}
	(21*\pontMeret,0)--++(0,\kisOsztas)
	(22*\pontMeret,0)--++(0,\kisOsztas)
	(23*\pontMeret,0)--++(0,\kisOsztas)
	(24*\pontMeret,0)--++(0,\kisOsztas)
	(25*\pontMeret,0)--++(0,\nagyOsztas) node[anchor=west]{$25$}
	(26*\pontMeret,0)--++(0,\kisOsztas)
	(27*\pontMeret,0)--++(0,\kisOsztas)
	(28*\pontMeret,0)--++(0,\kisOsztas)
	(29*\pontMeret,0)--++(0,\kisOsztas)
	(30*\pontMeret,0)--++(0,\nagyOsztas) node[anchor=west]{$30$}
	(31*\pontMeret,0)--++(0,\kisOsztas)
	(32*\pontMeret,0)--++(0,\kisOsztas)
	(33*\pontMeret,0)--++(0,\kisOsztas)
	(34*\pontMeret,0)--++(0,\kisOsztas)
	(35*\pontMeret,0)--++(0,\nagyOsztas) node[anchor=west]{$35$}
	(36*\pontMeret,0)--++(0,\kisOsztas)
	(37*\pontMeret,0)--++(0,\kisOsztas)
	(38*\pontMeret,0)--++(0,\kisOsztas)
	(39*\pontMeret,0)--++(0,\kisOsztas)
	(40*\pontMeret,0)--++(0,\nagyOsztas) node[anchor=west]{$40$};
	\draw[blue] (40*\pontMeret,0)--(46*\pontMeret,0)
	(41*\pontMeret,0)--++(0,\kisOsztas)
	(42*\pontMeret,0)--++(0,\kisOsztas)
	(43*\pontMeret,0)--++(0,\kisOsztas)
	(44*\pontMeret,0)--++(0,\kisOsztas)
	(45*\pontMeret,0)--++(0,\nagyOsztas) node[anchor=west]{$45$}
	(46*\pontMeret,0)--++(0,\kisOsztas);
	% féléves jegy ponthatárok
	\jegyMarker{0}{\nagyOsztas}{\jegyMagassag}{red}{$0$}{$1$}
	\jegyMarker{16*\pontMeret}{\kisOsztas}{\nagyOsztas-\kisOsztas+\jegyMagassag}{red}{$16$}{$2$}
	\jegyMarker{22*\pontMeret}{\kisOsztas}{\nagyOsztas-\kisOsztas+\jegyMagassag}{red}{$22$}{$3$}
	\jegyMarker{28*\pontMeret}{\kisOsztas}{\nagyOsztas-\kisOsztas+\jegyMagassag}{red}{$28$}{$4$}
	\jegyMarker{34*\pontMeret}{\kisOsztas}{\nagyOsztas-\kisOsztas+\jegyMagassag}{red}{$34$}{$5$}
	\draw[red] (0,0.5*\kisOsztas)--++(16*\pontMeret,0);
	\node at(0,\nagyOsztas+\jegyMagassag+0.2) [anchor=east,red] {Ponthatárok (alsó)};
	\node at(0,\nagyOsztas+\jegyMagassag-0.3) [anchor=east,red,draw,xshift=-1.5mm] {Féléves jegy};
	% NHZ tengely
	\draw (0,-\kisTengelyEltolas-\nagyTengelyEltolas) node[anchor=east]{\makecell[r]{{\color{red}Minimumkövetelmény}\\NHZ pont}} --++(20*\pontMeret,0)
	(0,-\kisTengelyEltolas-\nagyTengelyEltolas)--++(0,\nagyOsztas) node[anchor=west]{$0$}
	(\pontMeret,-\kisTengelyEltolas-\nagyTengelyEltolas)--++(0,\kisOsztas)
	(2*\pontMeret,-\kisTengelyEltolas-\nagyTengelyEltolas)--++(0,\kisOsztas)
	(3*\pontMeret,-\kisTengelyEltolas-\nagyTengelyEltolas)--++(0,\kisOsztas)
	(4*\pontMeret,-\kisTengelyEltolas-\nagyTengelyEltolas)--++(0,\kisOsztas)
	(5*\pontMeret,-\kisTengelyEltolas-\nagyTengelyEltolas)--++(0,\nagyOsztas) node[anchor=west]{$5$}
	(6*\pontMeret,-\kisTengelyEltolas-\nagyTengelyEltolas)--++(0,\kisOsztas)
	(7*\pontMeret,-\kisTengelyEltolas-\nagyTengelyEltolas)--++(0,\kisOsztas)
	(8*\pontMeret,-\kisTengelyEltolas-\nagyTengelyEltolas)--++(0,\kisOsztas)
	(9*\pontMeret,-\kisTengelyEltolas-\nagyTengelyEltolas)--++(0,\kisOsztas)
	(10*\pontMeret,-\kisTengelyEltolas-\nagyTengelyEltolas)--++(0,\nagyOsztas) node[anchor=west]{$10$}
	(11*\pontMeret,-\kisTengelyEltolas-\nagyTengelyEltolas)--++(0,\kisOsztas)
	(12*\pontMeret,-\kisTengelyEltolas-\nagyTengelyEltolas)--++(0,\kisOsztas)
	(13*\pontMeret,-\kisTengelyEltolas-\nagyTengelyEltolas)--++(0,\kisOsztas)
	(14*\pontMeret,-\kisTengelyEltolas-\nagyTengelyEltolas)--++(0,\kisOsztas)
	(15*\pontMeret,-\kisTengelyEltolas-\nagyTengelyEltolas)--++(0,\nagyOsztas) node[anchor=west]{$15$}
	(16*\pontMeret,-\kisTengelyEltolas-\nagyTengelyEltolas)--++(0,\kisOsztas)
	(17*\pontMeret,-\kisTengelyEltolas-\nagyTengelyEltolas)--++(0,\kisOsztas)
	(18*\pontMeret,-\kisTengelyEltolas-\nagyTengelyEltolas)--++(0,\kisOsztas)
	(19*\pontMeret,-\kisTengelyEltolas-\nagyTengelyEltolas)--++(0,\kisOsztas)
	(20*\pontMeret,-\kisTengelyEltolas-\nagyTengelyEltolas)--++(0,\nagyOsztas) node[anchor=east]{$20$};
	\draw [dashed] (0,-\kisTengelyEltolas-\nagyTengelyEltolas+\nagyOsztas)--(0,0)
	(20*\pontMeret,-\kisTengelyEltolas-\nagyTengelyEltolas+\nagyOsztas)--(20*\pontMeret,0);
	% NZH követelmény
	\pontMarker{8*\pontMeret}{-\kisTengelyEltolas-\nagyTengelyEltolas+\kisOsztas}{\nagyOsztas-\kisOsztas+\jegyMagassag}{red}{$8$}
	\draw[red] (0,-\kisTengelyEltolas-\nagyTengelyEltolas+0.5*\kisOsztas)--++(8*\pontMeret,0);
	% kHZ tengely
	\draw (20*\pontMeret,-\kisTengelyEltolas-2*\nagyTengelyEltolas) node[anchor=south east]{\makecell[r]{{\color{blue}Pluszpont követelmény}\\{\color{red}Minimumkövetelmény}\\kZH-k csökkenő elért pontszámmal}} node[anchor=north east,blue,draw,xshift=-1.5mm]{Szerzett pluszpont} --++(20*\pontMeret,0)
	(20*\pontMeret,-\kisTengelyEltolas-2*\nagyTengelyEltolas)--++(0,\nagyOsztas) node[anchor=west]{$0$}
	(21*\pontMeret,-\kisTengelyEltolas-2*\nagyTengelyEltolas)--++(0,\kisOsztas)
	(22*\pontMeret,-\kisTengelyEltolas-2*\nagyTengelyEltolas)--++(0,\kisOsztas)
	(23*\pontMeret,-\kisTengelyEltolas-2*\nagyTengelyEltolas)--++(0,\kisOsztas)
	(24*\pontMeret,-\kisTengelyEltolas-2*\nagyTengelyEltolas)--++(0,\nagyOsztas) node[anchor=east]{$4$} node[anchor=west]{$0$}
	(25*\pontMeret,-\kisTengelyEltolas-2*\nagyTengelyEltolas)--++(0,\kisOsztas) 
	(26*\pontMeret,-\kisTengelyEltolas-2*\nagyTengelyEltolas)--++(0,\kisOsztas)
	(27*\pontMeret,-\kisTengelyEltolas-2*\nagyTengelyEltolas)--++(0,\kisOsztas)
	(28*\pontMeret,-\kisTengelyEltolas-2*\nagyTengelyEltolas)--++(0,\nagyOsztas) node[anchor=east]{$4$} node[anchor=west]{$0$}
	(29*\pontMeret,-\kisTengelyEltolas-2*\nagyTengelyEltolas)--++(0,\kisOsztas)
	(30*\pontMeret,-\kisTengelyEltolas-2*\nagyTengelyEltolas)--++(0,\kisOsztas)
	(31*\pontMeret,-\kisTengelyEltolas-2*\nagyTengelyEltolas)--++(0,\kisOsztas)
	(32*\pontMeret,-\kisTengelyEltolas-2*\nagyTengelyEltolas)--++(0,\nagyOsztas) node[anchor=east]{$4$} node[anchor=west]{$0$}
	(33*\pontMeret,-\kisTengelyEltolas-2*\nagyTengelyEltolas)--++(0,\kisOsztas)
	(34*\pontMeret,-\kisTengelyEltolas-2*\nagyTengelyEltolas)--++(0,\kisOsztas)
	(35*\pontMeret,-\kisTengelyEltolas-2*\nagyTengelyEltolas)--++(0,\kisOsztas)
	(36*\pontMeret,-\kisTengelyEltolas-2*\nagyTengelyEltolas)--++(0,\nagyOsztas) node[anchor=east]{$4$} node[anchor=west]{$0$}
	(37*\pontMeret,-\kisTengelyEltolas-2*\nagyTengelyEltolas)--++(0,\kisOsztas)
	(38*\pontMeret,-\kisTengelyEltolas-2*\nagyTengelyEltolas)--++(0,\kisOsztas)
	(39*\pontMeret,-\kisTengelyEltolas-2*\nagyTengelyEltolas)--++(0,\kisOsztas)
	(40*\pontMeret,-\kisTengelyEltolas-2*\nagyTengelyEltolas)--++(0,\nagyOsztas) node[anchor=east]{$4$};
	\draw[dashed] (20*\pontMeret,-\kisTengelyEltolas-2*\nagyTengelyEltolas+\nagyOsztas)--(20*\pontMeret,-\kisTengelyEltolas-\nagyTengelyEltolas)
	(32*\pontMeret,-\kisTengelyEltolas-2*\nagyTengelyEltolas+\nagyOsztas)--++(0,\nagyTengelyEltolas)--(40*\pontMeret,-0.5)--++(0,0.5);
	\node at(26*\pontMeret,-\nagyTengelyEltolas){\LARGE$\times\frac{5}{3}$};
	% kZH követelmény
	\pontMarker{22*\pontMeret}{-\kisTengelyEltolas-2*\nagyTengelyEltolas+\kisOsztas}{\nagyOsztas-\kisOsztas+\jegyMagassag}{red}{$2$}
	\pontMarker{26*\pontMeret}{-\kisTengelyEltolas-2*\nagyTengelyEltolas+\kisOsztas}{\nagyOsztas-\kisOsztas+\jegyMagassag}{red}{$2$}
	\pontMarker{30*\pontMeret}{-\kisTengelyEltolas-2*\nagyTengelyEltolas+\kisOsztas}{\nagyOsztas-\kisOsztas+\jegyMagassag}{red}{$2$}
	\draw[red] (20*\pontMeret,-\kisTengelyEltolas-2*\nagyTengelyEltolas+0.5*\kisOsztas)--++(2*\pontMeret,0)
	(24*\pontMeret,-\kisTengelyEltolas-2*\nagyTengelyEltolas+0.5*\kisOsztas)--++(2*\pontMeret,0)
	(28*\pontMeret,-\kisTengelyEltolas-2*\nagyTengelyEltolas+0.5*\kisOsztas)--++(2*\pontMeret,0);
	\jegyMarkerAlt{34*\pontMeret}{-\kisTengelyEltolas-2*\nagyTengelyEltolas+\kisOsztas}{\nagyOsztas-\kisOsztas+\jegyMagassag}{blue}{$2$}{$+1$}
	\jegyMarkerAlt{38*\pontMeret}{-\kisTengelyEltolas-2*\nagyTengelyEltolas+\kisOsztas}{\nagyOsztas-\kisOsztas+\jegyMagassag}{blue}{$2$}{$+1$}
	% 16 minimumpont teljesítése
	\draw[darkgreen] (0,-0.5*\kisOsztas)--++(18*\pontMeret,0)
	(0,0)--++(0,-\kisOsztas)
	(8*\pontMeret,0)--++(0,-\kisOsztas)
	(18*\pontMeret,0)--++(0,-\kisOsztas);
	\node at(4*\pontMeret,-0.5*\kisOsztas)[anchor=north,darkgreen]{NZH};
	\node at(13*\pontMeret,-0.5*\kisOsztas)[anchor=north,darkgreen]{kZH$\times\frac{5}{3}$};
	% pluszpont tengely
	\draw[blue] (40*\pontMeret,-2*\kisTengelyEltolas-2*\nagyTengelyEltolas) node[anchor=south east]{Pluszpontok} --++(8*\pontMeret,0)
	(40*\pontMeret,-2*\kisTengelyEltolas-2*\nagyTengelyEltolas)--++(0,\nagyOsztas) node[anchor=west]{$0$}
	(41*\pontMeret,-2*\kisTengelyEltolas-2*\nagyTengelyEltolas)--++(0,\kisOsztas)
	(42*\pontMeret,-2*\kisTengelyEltolas-2*\nagyTengelyEltolas)--++(0,\kisOsztas)
	(43*\pontMeret,-2*\kisTengelyEltolas-2*\nagyTengelyEltolas)--++(0,\kisOsztas)
	(44*\pontMeret,-2*\kisTengelyEltolas-2*\nagyTengelyEltolas)--++(0,\kisOsztas) 
	(45*\pontMeret,-2*\kisTengelyEltolas-2*\nagyTengelyEltolas)--++(0,\kisOsztas) 
	(46*\pontMeret,-2*\kisTengelyEltolas-2*\nagyTengelyEltolas)--++(0,\nagyOsztas) node[anchor=east]{$6$}
	(47*\pontMeret,-2*\kisTengelyEltolas-2*\nagyTengelyEltolas)--++(0,\kisOsztas) 
	(48*\pontMeret,-2*\kisTengelyEltolas-2*\nagyTengelyEltolas)--++(0,\nagyOsztas) node[anchor=east]{$8$};
	\draw[blue,dashed] (40*\pontMeret,-2*\kisTengelyEltolas-2*\nagyTengelyEltolas+\nagyOsztas)--(40*\pontMeret,-\kisTengelyEltolas-2*\nagyTengelyEltolas)
	(40*\pontMeret,-\kisTengelyEltolas-2*\nagyTengelyEltolas+\nagyOsztas)--(40*\pontMeret,-0.5)
	(46*\pontMeret,-2*\kisTengelyEltolas-2*\nagyTengelyEltolas+\nagyOsztas)--(46*\pontMeret,0);
	% kZH pluszpont tengely
	\draw[blue] (40*\pontMeret,-3*\kisTengelyEltolas-2*\nagyTengelyEltolas) node[anchor=south east]{kZH-ból} --++(2*\pontMeret,0)
	(40*\pontMeret,-3*\kisTengelyEltolas-2*\nagyTengelyEltolas)--++(0,\nagyOsztas) node[anchor=west]{$0$}
	(41*\pontMeret,-3*\kisTengelyEltolas-2*\nagyTengelyEltolas)--++(0,\kisOsztas)
	(42*\pontMeret,-3*\kisTengelyEltolas-2*\nagyTengelyEltolas)--++(0,\nagyOsztas) node[anchor=west]{$2$};
	\draw[blue,dashed] (40*\pontMeret,-3*\kisTengelyEltolas-2*\nagyTengelyEltolas+\nagyOsztas)--(40*\pontMeret,-2*\kisTengelyEltolas-2*\nagyTengelyEltolas)
	(42*\pontMeret,-3*\kisTengelyEltolas-2*\nagyTengelyEltolas+\nagyOsztas)--(42*\pontMeret,-2*\kisTengelyEltolas-2*\nagyTengelyEltolas);
	% HF pluszpont tengely
		\draw[blue] (42*\pontMeret,-4*\kisTengelyEltolas-2*\nagyTengelyEltolas) node[anchor=south east]{HF-ből} --++(6*\pontMeret,0)
	(42*\pontMeret,-4*\kisTengelyEltolas-2*\nagyTengelyEltolas)--++(0,\nagyOsztas) node[anchor=west]{$0$}
	(43*\pontMeret,-4*\kisTengelyEltolas-2*\nagyTengelyEltolas)--++(0,\kisOsztas)
	(44*\pontMeret,-4*\kisTengelyEltolas-2*\nagyTengelyEltolas)--++(0,\kisOsztas)
	(45*\pontMeret,-4*\kisTengelyEltolas-2*\nagyTengelyEltolas)--++(0,\kisOsztas)
	(46*\pontMeret,-4*\kisTengelyEltolas-2*\nagyTengelyEltolas)--++(0,\kisOsztas)
	(47*\pontMeret,-4*\kisTengelyEltolas-2*\nagyTengelyEltolas)--++(0,\kisOsztas)
	(48*\pontMeret,-4*\kisTengelyEltolas-2*\nagyTengelyEltolas)--++(0,\nagyOsztas) node[anchor=east]{$6$};
	\draw[blue,dashed] (42*\pontMeret,-4*\kisTengelyEltolas-2*\nagyTengelyEltolas+\nagyOsztas)--(42*\pontMeret,-3*\kisTengelyEltolas-2*\nagyTengelyEltolas)
	(48*\pontMeret,-4*\kisTengelyEltolas-2*\nagyTengelyEltolas+\nagyOsztas)--(48*\pontMeret,-2*\kisTengelyEltolas-2*\nagyTengelyEltolas);
\end{tikzpicture}
			\caption{A követelményrendszer ábrázolva}
			\label{fig:pontrendszer}
		\end{figure}
	\end{landscape}
\end{document}